\documentclass{article}
\usepackage[cm]{fullpage}
\usepackage{color}
\usepackage{hyperref}

\hypersetup{breaklinks=true,%
pagecolor=white,%
colorlinks=true,%
linkcolor=cyan,%
urlcolor=MyDarkBlue}

\definecolor{MyDarkBlue}{rgb}{0,0.0,0.45}

%%%%%%%%%%%%%%%%%%%%%%%%%%
% Formatting parameters  %
%%%%%%%%%%%%%%%%%%%%%%%%%%

\newlength{\tabin}
\setlength{\tabin}{1em}
\newlength{\secsep}
\setlength{\secsep}{0.1cm}

\setlength{\parindent}{0in}
\setlength{\parskip}{0in}
\setlength{\itemsep}{0in}
\setlength{\topsep}{0in}
\setlength{\tabcolsep}{0in}

\definecolor{contactgray}{gray}{0.3}
\pagestyle{empty}

%%%%%%%%%%%%%%%%%%%%%%%%%%
%  Template Definitions  %
%%%%%%%%%%%%%%%%%%%%%%%%%%

\newcommand{\lineunder}{\vspace*{-8pt} \\ \hspace*{-6pt} \hrulefill \\ \vspace*{-15pt}}
\newcommand{\name}[1]{\begin{center}\textsc{\Huge#1}\\\end{center}}
\newcommand{\program}[1]{\begin{center}\textsc{#1}\end{center}}
\newcommand{\contact}[1]{\begin{center}\color{contactgray}{\small#1}\end{center}}

\newenvironment{tabbedsection}[1]{
  \begin{list}{}{
      \setlength{\itemsep}{0pt}
      \setlength{\labelsep}{0pt}
      \setlength{\labelwidth}{0pt}
      \setlength{\leftmargin}{\tabin}
      \setlength{\rightmargin}{\tabin}
      \setlength{\listparindent}{0pt}
      \setlength{\parsep}{0pt}
      \setlength{\parskip}{0pt}
      \setlength{\partopsep}{0pt}
      \setlength{\topsep}{#1}
    }
  \item[]
}{\end{list}}

\newenvironment{nospacetabbing}{
    \begin{tabbing}
}{\end{tabbing}\vspace{-1.2em}}

\newenvironment{resume_header}{}{\vspace{0pt}}


\newenvironment{resume_section}[1]{
  \filbreak
  \vspace{2\secsep}
  \textsc{\large#1}
  \lineunder
  \begin{tabbedsection}{\secsep}
}{\end{tabbedsection}}

\newenvironment{resume_subsection}[2][]{
  \textbf{#2} \hfill {\footnotesize #1} \hspace{2em}
  \begin{tabbedsection}{0.5\secsep}
}{\end{tabbedsection}}

\newenvironment{subitems}{
  \renewcommand{\labelitemi}{-}
  \begin{itemize}
      \setlength{\labelsep}{1em}
}{\end{itemize}}

\newenvironment{resume_employer}[4]{
  \vspace{\secsep}
  \textbf{#1} \\ 
  \indent {\small #2} \hfill {\footnotesize#3 (#4)}
  \begin{tabbedsection}{0pt}
  \begin{subitems}
}{\end{subitems}\end{tabbedsection}}


%%%%%%%%%%%%%%%%%%%%%%%%%%
%     Start Document     %
%%%%%%%%%%%%%%%%%%%%%%%%%%

\begin{document}

\begin{resume_header}
\name{Malcolm St. John}
\program{Génie informatique}
\contact{malcolm.st.john@usherbrooke.ca \hspace{2.5cm} 514-668-9136 \hspace{2cm}2387 Rue de Verdun, Sherbrooke Qc}
\end{resume_header}

\begin{resume_section}{Éducation}
  \begin{resume_subsection}[Sherbrooke, QC (2017--Présent)]{Université de Sherbrooke}
    \begin{subitems}
      \item Baccalauréat en Génie Informatique
    \end{subitems}
  \end{resume_subsection}
  
  \begin{resume_subsection}[Montréal, QC (2015--)]{Université McGill}
    \begin{subitems}
      \item Études universitaires en mathématiques (30 crédits)
     \end{subitems}
  \end{resume_subsection}
  
    \begin{resume_subsection}[Sherbrooke, QC (2012-2015)]{Cégep de Sherbrooke}
    \begin{subitems}
      \item DÉC en sciences de la nature et musique profil interprétation classique
      \item Finaliste national au concours de musique du Canada en 2013 et 2015      
      \item Prix Marie-Victorin pour excellence en mathématiques, 2015
     \end{subitems}
  \end{resume_subsection}
\end{resume_section}

\begin{resume_section}{Expériences Professionnelles}
  \begin{resume_employer}{Média5 Corporation}{Développeur low level; Bash, C}{Sherbrooke, QC}{Hiver 2020}
    \item Création et réparation de systèmes Linux personnalisés pour plusieurs architectures de CPU.
    \item Analyse de fichiers binaires désassemblés. Architecture de systèmes Linux. 
  \end{resume_employer}{}
  
  \begin{resume_employer}{CGI Inc.}{Développeur full stack; Java (Spring), Typescript (Angular), Microsoft SQL Server}{Sherbrooke, QC}{Été 2019}
    \item Conception et implémentation de endpoint d'API selon les demandes du client.
  \end{resume_employer}
  
    \begin{resume_employer}{Ubisoft Montréal}{Programmeur Généraliste TG/Pilot; C++}{Montreal, QC}{Automne 2018}
    \item Conception et implémentation d'un nouveau modèle de gestion des données d'instances de jeu sur le cloud.
    \item Analyse des statistiques de performance de génération d'environnements.
  \end{resume_employer}
  
%   \begin{resume_employer}{École de musique Pianissimo}{Professeur de piano et violon}{Sherbrooke, QC}{2014-2018}
%   \end{resume_employer}
\end{resume_section}

\begin{resume_section}{Projets}
 \begin{resume_subsection}[Mai 2020 - Décembre 2021]{Projet de fin de baccalauréat}
  \begin{subitems}
    \item Projet en association avec Ubisoft Montréal. Mon travail consiste à développer des microservices en Go, la création d'un pipeline CI/CD et j'ai aussi le rôle de scrum master.
    \end{subitems}
  \end{resume_subsection}
 
  \begin{resume_subsection}[Janvier 2020]{Conductify à ConUHacks}
  \begin{subitems}
    \item Développement d'une application qui fait des requêtes à l'api de Octave et utilise par la suite de l'apprentissage machine pour diviser la chanson en ses instruments dont le volume peut être ajusté par un capteur de mouvement sur un Arduino.
    \end{subitems}
  \end{resume_subsection}
  
  \begin{resume_subsection}[Hiver 2019]{Moniteur d'exercice pour vélo stationnaire}
  \begin{subitems}
    \item Mesurer le rythme cardiaque, la vitesse du pédalier, la vitesse de la roue et la déportation du genou avec des capteurs connectés à une carte Zybo Z7-10. 
    \end{subitems}
  \end{resume_subsection}

  \begin{resume_subsection}[Été 2018]{Conception d'un framework I/O}
  \begin{subitems}
    \item Conception d’un système d’entrées et sorties modulable qui permet à l’usager de spécifier ses besoins.
    \end{subitems}
  \end{resume_subsection}

  \begin{resume_subsection}[Hiver 2018]{Jeu de tetris contrôlé par la voix}
    \begin{subitems}
        \item Implémentation de la logique du jeu en C++.
    \end{subitems}
  \end{resume_subsection}
  
    \begin{resume_subsection}[2014-2015]{Recherche sur la pollution lumineuse}
        \begin{subitems}
        \item Développement et implémentation d'un nouveau modèle d'analyse de l'impact des nuages sur la pollution lumineuse urbaine avec le professeur Martin Aubé au Cégep de Sherbrooke.
        \end{subitems}
    \end{resume_subsection}
\end{resume_section}

\begin{resume_section}{Connaissances}
\begin{nospacetabbing}

  \textbf{Programmation}  \= C, C++, Python, Java, VHDL, Bash scripting, MIPS assembly, PostgreSQL\\*
  \textbf{Frameworks} \> Spring Boot, Flask\\*
  \textbf{Outils} \> Git, \LaTeX \\*
  \textbf{Langues} \> Anglais, Français\\*
\end{nospacetabbing}
\end{resume_section}

% \begin{resume_section}{Connaissances et accomplissements}
%   \begin{nospacetabbing}

%   \textbf{Connaissances techniques}  \= C/C++, Python, Java, Unix/Linux, SQL, Git, VHDL, \LaTeX\\*
%   \textbf{Langues} \> Anglais et Français\\*
%   \textbf{Mathémathiques} \> Prix Marie-Victorin pour excellence en mathématiques 2015, 2 participations au Putnam\\*
%   \textbf{Intérêts} \> Mathématiques, Gastronomie, Jeux de cartes, Piano classique, Noyeau Linux\\*
%   \end{nospacetabbing}
% \end{resume_section}

\end{document}
\textbf{}